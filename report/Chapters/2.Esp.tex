\chapter{Espíritus animales}

\noindent Esta investigación es un esfuerzo por intentar afianzar el vínculo entre los fundamentos microeconómicos y los fenómenos macroeconómicos: entre la mente de los individuos y la innovación en los mercados. 

\newpage

\section{Conducta e instituciones}

\subsection{Cambios cerebrales y genéticos}

\noindent Brizendine (2010) escribe que algunos científicos piensan que ciertas áreas del cerebro son como centros de actividad que mandan señales eléctricas a otras áreas del cerebro ocasionando un determinado comportamiento.\footnote{ Por ejemplo, en el hombre la corteza del cíngulo anterior pesa opciones, detecta conflicto y motiva decisiones. La unión temporoparietal busca soluciones rápidas y ante situaciones estresantes toma en cuenta la perspectiva de otros individuos. La corteza cingulada anterior rostral se encarga de procesar los errores sociales, como la aprobación o desaprobación de otros.}

\begin{quote}
    \small{Mientras que la distinción entre los cerebros de niños y niñas empieza biológicamente, estudios recientes muestran que es \textit{solo} el comienzo. La estructura cerebral no está escrita sobre piedra en el nacimiento ni al final de la infancia, como antes se creía, sino que continúa cambiando a lo largo de la vida. Más que ser inmutable, nuestros cerebros son mucho más plásticos y cambiables de lo que los científicos creían hace una década. El cerebro humano es también la máquina de aprendizaje más talentosa que conocemos. Así que nuestra cultura y el cómo nos enseñaron a comportarnos desempeñan un papel importante en el diseño y reestructura de nuestros cerebros (Brizendine 2010, 5-6).}
\end{quote}

% Para citas muy largas es mejor el \begin{quote}

\vspace{1em}
\noindent \textbf{Hipótesis 3.} \hfill\begin{minipage}{\dimexpr\textwidth-3cm}
\textit{La intensidad religiosa está relacionada negativamente con la innovación.}
\end{minipage}
\vspace{1em}

% Para plantear hipótesis

\begin{table}[H]
\centering
\caption{Índices de modernidad y tradicionalismo}
\label{PHEL}
\begin{tabular}{|ccc|}
\hline
 País & Índice de modernidad & Índice de tradicionalismo  \\ 
\hline
Alemania & 0.58 & 0.45 \\
Austria & 0.55 & 0.49  \\
Bélgica & 0.50 & 0.49  \\
Canadá & 0.61 & 0.50 \\
Dinamarca & 0.58 & 0.44 \\ 
España & 0.47 & 0.62 \\
Estados Unidos & 0.59 & 0.44  \\
Finlandia & 0.62 & 0.38 \\
Francia & 0.49 & 0.59 \\ 
Holanda & 0.58 & 0.49 \\ 
Irlanda & 0.54 & 0.59  \\
Islandia & 0.63 & 0.54 \\
Italia & 0.56 & 0.58  \\
Japón & 0.42 & 0.48 \\
Noruega & 0.53 & 0.44 \\
Portugal & 0.50 & 0.71 \\ 
Reino Unido & 0.56 & 0.54  \\
Suecia & 0.62 & 0.51 \\
Promedio & 0.58 & 0.51 \\
\hline
\end{tabular}

\begin{tabular}{c}
\footnotesize{Fuente: Bojilov y Phelps (2012).}
\end{tabular}

\end{table}

% Para diseñar tablas
